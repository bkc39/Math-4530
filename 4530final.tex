\documentclass{article}
\usepackage[tmargin=1in,bmargin=1in,lmargin=1.5in,rmargin=1.5in]{geometry}
\usepackage{amsfonts,amsmath,amssymb,amsthm}
\usepackage{relsize,fancyhdr,parskip}
\usepackage{graphicx}
\usepackage[all,knot]{xy}

\pagestyle{fancy}
\lhead{Ben Carriel}
\chead{Math 4530 Final Exam}
\rhead{\today}

\parskip 7.2pt
\parindent 8pt

\DeclareMathOperator{\N}{\mathbb{N}}
\DeclareMathOperator{\Z}{\mathbb{Z}}
\DeclareMathOperator{\Q}{\mathbb{Q}}
\DeclareMathOperator{\R}{\mathbb{R}}
\DeclareMathOperator{\C}{\mathbb{C}}
\DeclareMathOperator{\capchi}{\raisebox{2pt}{$\mathlarger{\mathlarger{\chi}}$}}

\DeclareMathOperator{\divides}{\mathrel{|}}
\DeclareMathOperator{\suchthat}{\mathrel{:}}

\DeclareMathOperator{\lra}{\longrightarrow}
\DeclareMathOperator{\into}{\hookrightarrow}
\DeclareMathOperator{\onto}{\twoheadrightarrow}
\DeclareMathOperator{\bijection}{\leftrightarrow}

\newcommand{\problem}[1]{\noindent{\textbf{Problem #1}}\\}
\newcommand{\problempart}[1]{\noindent{\textbf{(#1)}}}
\newcommand{\exercise}[1]{\noindent{\textbf{Exercise #1:}}}

\newcommand{\der}[2]{\frac{\partial #1}{\partial #2}}
\newcommand{\norm}[1]{\|#1\|}
\newcommand{\diam}[1]{\text{diam}(#1)}
\newcommand{\seq}[2]{\{#1_{#2}\}_{#2 = 1}^\infty}

\DeclareMathOperator{\im}{\text{im}}

\newtheorem*{thm}{\\ Theorem}
\newtheorem*{lem}{\\ Lemma}
\newtheorem*{claim}{\\ Claim}
\newtheorem*{defn}{\\ Definition}
\newtheorem*{prop}{\\ Proposition}

\xyoption{arc}

\begin{document}
\section{Section 38: The Stone-\u{C}ech Compactification}

\exercise{38.4}

We are given a compactification $Y$ of the completely
regular space $X$. Let $c: X \to Y$ be the imbedding of $X$ into $Y$ and
let $\beta: X \to \beta(X)$ be the imbedding from $X$ into its Stone-\u{C}ech
Compactification. By the definition we have that $c$ extends uniquely
to a continuous map $g: \beta(X) \to Y$ as seen below
\[
\begin{xy}
(0,0)*+{\beta(X)}="BETAX"; (15,15)*+{Y}="Y"; (0,15)*+{X}="X";
{\ar_{\mathlarger{\beta}} "X";"BETAX"};
{\ar^{\mathlarger{c}} "X";"Y"};
{\ar_{\mathlarger{g}} "BETAX";"Y"};
\end{xy}
\]
It is clear that $g$ is the identity on $X$ because $c$ is the
identity on $X$ and $g$ extends $c$. To get that $g$ is a closed map
we need the following

\begin{lem}[Exercise 26.4]
  A continuous map $f: X \to Y$ from the compact space $X$ to the
  Hausdorff space $Y$ is closed.
\end{lem}
\begin{proof}
  Let $A$ be a closed subset of $X$. Then by Theorem 26.2 (Munkres
  pg. 165), $A$ is compact. We then apply Theorem 26.5 (Munkres
  pg. 166) to get that $f(A)$ is compact as well. Then because $Y$ is
  Hausdorff we can apply Theorem 26.3 (Munkres pg. 165) to get that
  $A$ is closed as well. Hence, $f$ is a closed map.
\end{proof}

We simply apply the above lemma to $g$ using the fact that $X$ is
compact and $Y$ is Hausdorff by construction. Finally, to get
surjectivity we note that the image of $g$ is dense in $Y$ because $g$
extends $c$ and we know that $\overline{c(X)} = \overline{X} = Y$ by
definition. Because $\overline{c(X)} \subset \beta(X)$ by
construction, and $g$ is a closed map that extends $c$, we see that
$g(\overline{X})$ is a closed set containing $X$ and therefore must
contain $\overline{X} = Y$. Hence, it contains all of $Y$ and we are
done.

\exercise{38.5}
\begin{enumerate}
\item [\textbf{(a)}] Let $f: S_\Omega \to \R$ be continuous. Following
  the hint, we fix an $\epsilon > 0$. We need to show that there is
  some $\alpha \in S_\Omega$ such that $|f(\beta) - f(\alpha)| <
  \epsilon$ for all $\beta > \alpha$. Suppose not, then we can find an
  increasing sequence $\{s_n\}_{n=1}^\infty$ in $S_\Omega$ such that
  $|f(s_n) - f(s_{n-1})| \geq \epsilon$. We then note that the
  sequence $\{s_n\}_{n=1}^\infty$ must converge to its least upper
  bound in $S_\Omega$, but the sequence $\{f(s_n)\}_{n=1}^\infty$ in
  $\R$ does not converge by construction. This contradicts the
  continuity of $f$ and so we must be able to find such an $\alpha$.

  Now we use the preceding to find that $f$ must be constant. Indeed,
  for each integer $n$ we know there is some element $\alpha_n$ such
  that $|f(\beta) - f(\alpha_n)| < 1/n$ by the above argument. This
  forms a sequnce $\{\alpha_n\}_{n=1}^\infty$ in $S_\Omega$ and let
  $\alpha$ be an upper bound. Then we have that $|f(\beta) -
  f(\alpha)| < 1/n$ for all $\beta > \alpha$ and every $n$. Hence,
  $|f(\beta) - f(\alpha)| = 0$ and so $f$ is constant on $(\alpha,
  \Omega)$ as desired.
\item [\textbf{(b)}] By the preceeding part, we know that any
  continuous function $S_\Omega \to \R$ is eventually constant and
  hence, is bounded. We then have the following
  \begin{claim}
    Any continuous, bounded function $f: S_\Omega \to \R$ extends uniquely to
    the one-point compactification $\overline{S_\Omega}$.
  \end{claim}
  \begin{proof}
    It is clear that boundedness is necessary because the extension of
    $f$ will send $S_\Omega$ into $\R$ and will therefore be
    bounded. Let $\bar{f}$ be the extension of $f$ to $S_\Omega$. The
    condition that $\bar{f}$ must satisfy is that for any sequence
    $\{s_n\}_{n=1}^\infty$ such that $s_n \to s$
    \[
    \lim_{n\to\infty} \bar{f}(s_n) = \bar{f}(s)
    \]
    This is clear when $s \in S_\Omega$ because $f$ is continuous on
    $S_\Omega$. If $s_n \to \overline{\Omega}$ then we note that $\R$
    has the least upper bound property and define
    $\bar{f}(\overline{\Omega}) = \sup f(s_n)$. This mapping is
    well-defined because $f$ is continuous and so if $\seq{t}{n}$ was
    another sequence converging to $\overline{\Omega}$ then $|f(s_n) -
    f(t_n)| < \epsilon$ for any $\epsilon > 0$ and $n > N$ for a
    sufficiently large $N$. This shows that $\bar{f}$ is well-defined
    and moreover, unique.
  \end{proof}
  Because every continuous function on $S_\Omega$ extends to the one
  point compactification we can apply Theorem 38.5 to the one-point
  compactification and $\beta(S_\Omega)$ to get that the two
  compactifications are equivalent.
\item [\textbf{(c)}] To see that any other compactification $Y$ of
  $S_\Omega$ is equivalent to $\overline{S_\Omega}$ we appeal to the
  previous exercise and note that $Y$ must be a quotient of
  $\beta(S_\Omega)$, and hence for each $f$ from $S_\Omega$ to $\R$
  let the extension $\bar{f}_{Y}$ be the restriction of $\bar{f}$ to
  $Y$. Then $\bar{f}_Y$ is uniquely determined and continuous because
  it is the composition of continuous functions. We then apply Theorem
  38.5 again to see that $Y$ is equivalent to
  $\overline{S_\Omega}$. Hence, any two compactifications of
  $S_\Omega$ are equivalent.
\end{enumerate}

\exercise{38.7}
\begin{enumerate}
\item [\textbf{(a)}] Consider the function $f: X \to \{0,1\}$ given by
  \[
  f(x) =
  \begin{cases}
    0 & x \in A \\
    1 & x \in X - A
  \end{cases}
  \]
  This function is continuous when $X$ has the discrete topology. As
  such, it has an extension $\bar{f}: \beta(X) \to \{0,1\}$ that is
  also continuous. Suppose that $\overline{A} \cap \overline{X-A} \neq
  \emptyset$, where closures are taken in $\beta(X)$. Then we would
  have some point $y$ which belongs to both. Then any neighborhood $V$
  of $y$ intersects both $A$ and $X-A$. This would mean that if
  $\bar{f}$ must be the constant function because otherwise there
  would be a discontinuity at $y$. But this is impossible because
  $\bar{f}$ extends $f$ and maps $A$ and $X-A$ to different
  values. Hence, $\overline{A}$ and $\overline{X-A}$ are disjoint.
\item [\textbf{(b)}] The first thing to notice is the following
  \begin{lem}
    Let $A$ and $X$ be as in the preceding part. Then $\overline{A}$
    is both open and closed in $\beta(X)$.
  \end{lem}
  \begin{proof}
    We note that the conlusion of the previous problem is the same as
    saying that $\overline{X-A} \subset \beta(X) - \overline{A}$. We
    then have the chain of inclusions
    \begin{align*}
      \beta(X) - \overline{A} &= \overline{X} - \overline{A} \\
      &\subseteq \overline{X-A} \\
      &\subseteq \beta(X) - \overline{A}
    \end{align*}
    So we must have that $\beta(X) - \overline{A} = \overline{X-A}$
    and thus $\overline{A}$ is open as well as closed.
  \end{proof}
  The plan is to use this fact to show that $\overline{U}$ is open (it
  is clearly closed). Indeed, we verify the following
  \begin{claim}
    If $U \in \beta(X)$ is open then $\overline{U\cap X} = \overline{U}$.
  \end{claim}
  \begin{proof}
    One inclusion is obvious, namely $U \cap X \subset U$ implies that
    $\overline{U \cap X} \subset \overline{U}$. For the reverse
    inclusion let $u$ be point of $\overline{U}$ and let $V$ be a
    neighborhood of $u$. Then we see that $V \cap U \neq \emptyset$
    because $u \in \overline{U}$. Consequently,
    \[
    (V \cap U) \cap X = V \cap (U\cap X) \neq \emptyset
    \]
    because $X$ is dense. So $x \in \overline{U\cap X}$, which gives
    the reverse inclusion so we are done.
  \end{proof}
  With the clain in hand we apply the lemma to the set $U\cap X$ to
  get that $\overline{U\cap X} = \overline{U}$ is both open and
  closed.
\item [\textbf{(c)}] Let $T$ be a subset of $\beta{X}$ containing at
  least two points $x$ and $y$. We need to show that $T$ is not
  connected. If we pick any open $U \subset \beta(X)$ such that $x \in
  U$ and $y \not\in U$, and such a choice is possible because
  $\beta(x)$ is Hausdorff, then we have that
  \[
  T = (T \cap \overline{U})\cup (T - \overline{U})
  \]
  is a separation of $T$. Thus, we have that the connected components
  of $\beta{X}$ are singletons, because they are both open and
  connected and any set larger than a singleton is not
  connected. Thus, $\beta(X)$ is totally disconnected.
\end{enumerate}

\exercise{38.9}
\begin{enumerate}
\item [\textbf{(a)}] We will suppose to the contrary and pursue a
  contradiction. The only natural thing to do is to find an
  ``inclusion-map'' test like we did in the previous exercise, and use
  the lift of that map to find a contradiction. So we need to find
  suitable sets $A,B$ such that
  \[
  \xymatrix{
    p_1 & \bullet & p_2
    \bullet & p_0 & \bullet
    p_3 & \bullet & p_4
  }
  \]
  $\bar{f}^{-1}(0)$ and $\bar{f}^{-1}(1)$ are disjoint, but map to the
  same value under $f$, and then we will have our contradiction.

  Indeed, suppose that $\seq{x}{n}$ is a sequence in $X$ that
  converges to $x \in \beta(X) - X$. We need to construct disjoint
  sets such that $x_n \to x$ in both $A$ and $B$. One way to do this
  is to remove the duplicates from $x_n$. We create the new sequence
  $x'_n$ from $x_n$ inductively by setting $x'_1 = x_1$ and $x_n =
  \inf_m \{x_m \mathrel{|} x_m \neq x'_k, k < n\}$. It is easy to see
  that $x'_n \to x$ because any neighborhood of $x$ contains all but
  finitely many of the $x_n$, and let $x_N$ be the highest index not
  contained in that neighborhood. Then we have that for sufficiently
  large $M$, $x'_m = x_n$ with $m > M, n > N$ by construction. So
  $x'_n \to x$ as well.

  By construction, we have that the sets $A = \{x'_1, x'_3, x'_5,
  \ldots\}$ and $B = \{x'_2, x'_4, x'_6,\ldots\}$ are disjoint, and
  have limit point $x$. We then verify the following
  \begin{claim}
    We have $\overline{A} = A \cup \{x\}$ and $\overline{B} = B \cup
    \{x\}$, where the closures are taken in $\beta(X)$.
  \end{claim}
  \begin{proof}
    We have already shown that $x$ is a limit point of $A$ and hence,
    $x \in \overline{A}$. Since $A \subset A \cup \{x\} \subseteq
    \overline{A}$ and $\overline{A}$ is the smallest closed set
    containing $A$ it will suffice to show that $A \cup \{x\}$ is
    closed.

    Consider $C = \beta(X) - (A \cup \{x\})$. If we choose any point
    $y \in C$ then $y$ is not a limit of the sequence
    $\{x_{2n+1}\}_{n=1}^\infty$ by construction. Because $y$ is not a limit of
    the sequence we an find a neighborhood $V$ of $y$ that contains
    only finitely many of the $x_{2n+1}$. Because $y$ is not in $A$ we
    can remove these finitley many points from $V$ to get a new
    neighborhood $V'$ such that $V' \cap (A \cup \{x\}) =
    \emptyset$. Hence, $V'$ is wholly contained in $C$ and so $C$ is
    open and as a result $A \cup \{x\}$ is closed.

    The proof for $B \cup \{x\}$ is analogous.
  \end{proof}
  This shows the first property that we wanted of $A$ and $B$, namely
  that $\overline{A}\cap\overline{B} \neq \emptyset$ while $A\cap B =
  \emptyset$.

  Now we make the observation that $A$ and $B$ are both closed in $X$
  because we have that
  \[
  \overline{A} = X \cap \overline{A} = X \cap (A \cup \{x\}) = A
  \]
  and similarly for $B$. Then we can apply the Urysohn lemma,
  recalling that $X$ is normal and $A,B$ are disjoint closed subsets
  of $X$. Hence, $A$ and $B$ can be separated by a continuous function
  $f: X \to [0,1]$. We then lift $f$ to a function $\bar{f}: \beta(X)
  \to [0,1]$ that agrees with $f$ on $X$. We then note that $A \subset
  f^{-1}(0)$ and $B \subset f^{-1}(1)$ by construction so that
  $\overline{A} \subset \bar{f}^{-1}(0)$ and $\overline{1} \subset
  \bar{f}^{-1}(1)$. So $\overline{A}$ and $\overline{B}$ are disjoint.

  This gives a contradiction in that $\overline{A} \cap \overline{B}$
  is both empty and non-empty and so no point of $\beta(X) - X$ can be
  a limit of a sequence of points in $X$.
\item [\textbf{(b)}] Suppose that $X$ is completely regular and
  noncompact. Then $X$ must be properly contained in $\beta(X)$
  because $\beta(X)$ is compact and $X$ is not. Because $X$ is
  completely regular we can apply the previous part of this exercise
  to see that no point in $\beta(X) - X$ is a limit of a sequence in
  $X$. However, because one-point sets are closed in $X$ there is a
  function that separates every point of $\beta(X) - X$ and $X$. If we
  extend such a function to $\beta(X)$ we see that $\beta(X)$ does
  not satisfy Theorem 21.3 (Munkres pg. 130), and is therefore not
  metrizable.
\end{enumerate}

\exercise{38.10}
\begin{enumerate}
\item[\textbf{(i)}] Let $1_X$ be the identity map of $X$. Then if we
  consider the canonical imbedding of $X$ into $\beta(X)$, which is
  equal to $1_X$ on $X$, we can lift this function to a unique continuous
  function $\bar{f}: \beta(X) \to \beta(X)$ that is the identity on
  $X$. Because $X$ is dense in $\beta(X)$ and $\beta(X)$ is compact
  and Hausdorff we have that $\bar{f}$ determined by its values on $X$
  and therefore must be the identity on all of $\beta(X)$.
\item[\textbf{(ii)}] This is a repeated application of the universal
  property of $\beta(X)$. The situation is most readily understood
  with a diagram
  \[
  \begin{xy}
    (0,0)*+{\beta(Y)}="BETAY"; (0,15)*+{Y}="Y";
    (-15,0)*+{\beta(X)}="BETAX"; (-15,15)*+{X}="X";
    (15,15)*+{Z}="Z";
    {\ar "X";"BETAX"};
    {\ar "Y";"BETAY"};
    {\ar^{\mathlarger{f}} "X";"Y"};
    {\ar^{\mathlarger{g}} "Y";"Z"};
    {\ar_{\mathlarger{\bar{f}}} "BETAX";"Y"};
    {\ar_{\mathlarger{\bar{g}}} "BETAY";"Z"};
    {\ar "BETAX";"BETAY"};
  \end{xy}
  \]
  We simply apply the lifting twice. We first are given a function $f:
  X \to Y$, which can be uniquely extended to the function $\bar{f}:
  \beta(X) \to Y$ and so $g \circ \bar{f}$ is an extension of $g\circ
  f$ from $\beta(X) \to Z$. However, $g$ can be lifted to a map
  $\bar{g}: \beta(Y) \to Z$ and so $g\circ \bar{f}$ extends to the map
  $\bar{g} \circ \bar{f}$ and so because all extensions are unique we have
  \[
  \beta(g \circ f) = \beta(g) \circ \beta(f)
  \]
  In other words, the diagram above commutes.
\end{enumerate}

\section{Section 43: Complete Metric Spaces}
\noindent{\textbf{Exercise 43.5} (Banach Contraction Principle)\textbf{:}}

Consider any point $x_0 \in X$. From $x_0$ we construct the sequence
$x_0, f(x_0), f(f(x_0)), f^3(x_0),\ldots$, call it $\seq{x}{n}$. We
first show that this sequence converges to some $x \in X$, which we
hope will be our fixed point. Let $D = d(x_0, x_1)$, so because $f$ is
a contraction $d(x_2,x_1) \leq \alpha D$ and in general we can estimate
\begin{align*}
  d(x_n,x_m) &\leq \sum_{k=n}^m d(x_k, d_{k+1}) \\
  &= \sum_{k=1}^m d(x_k, d_{k+1}) - \sum_{k=1}^n d(x_k, d_{k+1}) \\
  &= \frac{D}{1-\alpha} (\alpha^{n+1} - \alpha^{m+1}) \\
  &\leq \alpha^{n+1}\frac{D}{1-\alpha}
\end{align*}
This quantity goes to $0$ as $n$ gets large because $\alpha <
1$. Consequently, we have that for sufficiently large $n,m$ that
$d(x_n, x_m) < \epsilon$ for any $\epsilon > 0$ and so $\seq{x}{n}$ is
a Cauchy sequence. Because $X$ is a complete metric space, we have
that $\seq{x}{n}$ converges in $X$, say to $x$.

We will now show that $x$ is the desired fixed point. Note that
\[
\lim_{n\to\infty} d(f(x_n),x_n) = \lim_{n\to\infty} \alpha^n D = 0
\]
However, because $d$ is continuous on $X$ that we can exchange the
limiting operation (twice) and get that $d(f(x),x) = 0$ so $x = f(x)$
and $x$ is the desired fixed point.

For uniqueness, we suppose that there were another point $y$ such that
$f(y) = y$. If $y\neq x$ then we have that $d(x,y) > 0$ and so we
apply $f$ to get that
\[
d(f(x), f(y)) \leq \alpha d(x,y) < d(x,y)
\]
But then either $f(x) \neq x$ or $f(y) \neq y$ which is a
contradiction. So $x$ must be the unique fixed point.
\hfill\qedsymbol

\exercise{43.9}
\begin{enumerate}
\item [\textbf{(a)}] To see that $\sim$ is an equivalence relation we
  need to verify that it is reflexive, symmetric and
  transitive. Reflexivity is clear because $d(x_n,x_n) = 0$ for all
  $n$ and so it clearly converges to 0. Symmetry is also evident from
  the symmetry of the metric. If we have that $\textbf{x} \sim
  \textbf{y}$ then we have that
  \[
  d(x_n,y_n) = d(y_n,x_n) = 0
  \]
  Which gives that $\textbf{y} \sim \textbf{x}$. Transitivity follows
  from the triangle inequality. Fix an $\epsilon > 0$ and suppose that
  $\textbf{x} \sim \textbf{y}$ and $\textbf{y} \sim \textbf{z}$. Then
  we see that
  \[
    d(x_n,z_n) \leq d(x_n,y_n) + d(y_n,z_n) \\
  \]
  So that when we let $n \to \infty$ we get that $d(x_n,z_n) \to
  0$. So $\sim$ is an equivalence relation.
\item [\textbf{(b)}] Let $x,y$ be two elements of $X$. We need to show
  that the map $h$ is injective, continuous and preserves the
  metric. Indeed, we see that $h$ is injective because if $h(x) =
  h(y)$ then $[(x,x,\ldots)] = [(y,y,\ldots)]$ which clearly implies
  that $x = y$ because otherwise the two sequences would differ in
  every term.

  To see that $f$ is continuous we fix an $\epsilon > 0$ and note that
  if we set $\delta = \epsilon$ then $d(h(x),h(y)) < \epsilon$
  whenever $d(x,y) < \delta$ and so $f$ is continuous.

  To see that $h$ preserves the metric observe that
  \[
  d(x,y) = \lim_{n\to\infty} d(x,y) = D([(x,x,\ldots)], [(y,y,\ldots)])
  \]
  Because the limit of a constant sequence is just that constant.
\item [\textbf{(c)}] Now suppose that we are given an $\textbf{x} =
  (x_1,x_2\ldots) \in Y$. We need to show that there is a sequence of
  points in $Y$ converging to $\textbf{x}$. Indeed, let $\epsilon > 0$
  be given and recall that $x_1,x_2\ldots$ is a Cauchy sequence in $Y$
  because it is in the image of $X$, which is defined as a set of
  Cauchy sequences. We then know that there is some $N$ such that $m,n
  > N$ implies that $d(x_mx_n) < \epsilon/2$. Set $\textbf{y} =
  h(x_N)$ then we note that
  \[
  D(\textbf{x},\textbf{y}) = \lim_{n\to\infty} d(x_n, x_N) \leq \epsilon/2
  \]
  So that $D(\textbf{x}, h(x_n)) \leq D(\textbf{x}, h(x_N)) +
  D(h(x_N),h(x_n)) < \epsilon$. So that $h(x_n) \to \textbf{x}$ and
  hence $X$ is dense in $Y$.
\item [\textbf{(d)}] We are given that every Cauchy sequence in $A$
  converges in $(Z,\rho)$. Suppose that $\seq{z}{n}$ is a Cauchy
  sequence in $Z$. We need to prove that it converges as well. Well,
  to each $z_k$ we can find an $a_k \in A$ such that $\rho(a_k,z_k)
  \leq \epsilon/3$ for any choice of $\epsilon > 0$. Then we can see
  that $\seq{a}{n}$ is Cauchy because for sufficiently large $n,m$ we
  have that
  \[
  \rho(a_n,a_m) \leq \rho(a_n,z_n) + \rho(z_n, z_m) + \rho(z_m,a_m)
  \leq 3(\epsilon/3)
  \]
  Whenever $n,m$ are chosen large enough that $\rho(z_n,z_m) \leq
  \epsilon/3$. So $a_k$ is a Cauchy sequence as well. As a result, it
  must converge in $Z$, say to $a$. We will see that $z_n \to a$ as
  well. Indeed we use the triangle inequality to see that
  \[
  \rho(z_n,a) \leq \rho(z_n,a_n) + \rho(a_n,a)
  \]
  We can pick $n$ large enough that $\rho(a_n,a) \leq \epsilon/2$ and
  the result follows.
\item [\textbf{(e)}] By the previous part, it suffices to show that
  every Cauchy sequence in the dense subspace $h(X)$ converges in
  $Y$. Let $\textbf{y}_k$ be a Cauchy sequence in $h(X)$, where each
  $\textbf{y}_k = (y_k, y_k, \ldots)$. Since $h$ is an isometry we see
  that $d(y_n,y_m) = D(\textbf{y}_n, \textbf{y}_m)$ for each pair
  $n,m$. Hence, we see that $(y_1,y_2,\ldots)$ is a Cauchy sequence in
  $X$. If we set $\textbf{y}^* = (y_1,y_2,\ldots)$ then we want to
  show that $\textbf{y}_n \to \textbf{y}^*$. Choose $\epsilon > 0$ and
  note that there is an $N$ such that $d(y_n,y_m) < \epsilon/2$ for
  any $n,m > N$. So for each $m \geq N$ we have that
  \[
  D(\textbf{y}_m, \textbf{y}^*) = \lim_{n\to\infty} d(y_m,y_n) \leq \epsilon/2
  \]
  So we see that $\textbf{y} \to \textbf{y}*$ and we are done.
\end{enumerate}

\exercise{43.10}
\begin{thm}[Uniqueness of the Completion]
  Let $h: X \to Y$ and $h':X \to Y'$ be isometric imbeddings of the
  complete metric space $(X,d)$ in the complete metric spaces $(Y,D)$
  and $(Y',D')$, respectively. Then there is an isometry of
  $(\overline{(h(X)}, D)$ with$(\overline{(h'(X)}, D')$ that equals
  $h'h^{-1}$ on the subspace $h(X)$.
\end{thm}
\begin{proof}
  This is another situation that is made clear with a diagram
  \[
  \begin{xy}
    (0,0)*+{X}="X"; (-15,-15)*+{Y}="Y";
    (15,-15)*+{Y'}="Y'";
    {\ar_{\mathlarger{h}} "X";"Y"};
    {\ar^{\mathlarger{h'}} "X";"Y'"};
    {\ar@{<.>} "Y";"Y'"};
  \end{xy}
  \]
  We need to ``connect the dots'', so to speak. We can construct an
  isometry from the image of $h(X) \subset Y$ to the image $h'(X)
  \subset Y'$ by composing the maps above, namely, $h'\circ h^{-1}$ is
  the desired isometry. It is clear that this is an isometry because
  $h,h'$ are isometries and hence, continuous, injective, and distance
  preserving, which are all properties preserved under composition.

  Then we use the fact that $h(X)$ and $h'(X)$ are dense in $Y$ and
  $Y'$, respectively to extend $h'\circ h^{-1}$ to some continuous map
  $\varphi: Y \to Y'$ that agrees with $h'\circ h^{-1}$ on $h(X)$ by
  defining $\varphi(x) = \lim_{n\to\infty} \varphi(x_n)$ whenever $x_n
  \to x$ in $Y$. This sequence converges in $Y'$ because it is Cauchy
  and $Y'$ is complete. Because limits are unique in metric spaces,
  the map $\varphi$ is well-defined.

  We now need to verify that $\varphi$ is an isometry. To see that
  $\varphi$ is surjective we note that if $y \in Y'$ if we let $u_n$
  be a sequence in $h'(X)$ with limit $y$ then we consider the
  sequence of preimages $\varphi^{-1}(u_n)$ in $Y$. Because the space
  $Y'$ is Hausdorff we must have that
  \[
  \lim_{n\to\infty}\varphi(\varphi^{-1}(u_n)) = \lim_{n\to\infty} u_n = u
  \]
  as desired.

  For injectivity suppose that $\lim_{n\to\infty} \varphi(u_n) =
  \lim_{n\to\infty} \varphi(v_n)$ and $\lim_{n\to\infty} u_n = u$ and
  $\lim_{n\to\infty} v_n = v$. Fix an $\epsilon > 0$ and choose $M$
  large enough that $D'(\varphi(u_n), \varphi(v_n)) < \epsilon$ for all
  $n > M$. Then we have that
  \[
  D(u_n, v_n) = D'(\varphi(u_n), \varphi(v_n)) \leq \epsilon
  \]
  since $x$ is Hausdorff we conclude that $u = v$ and so $\varphi$ is
  injective.

  Finally, we note that the map $\varphi: \overline{h(X)} \to
  \overline{h'(X)}$ is continuous, because it is the continuous
  extension of $h' \circ h^{-1}$ and moreover it is bijective, and
  isometric, and thus is the desired function.
\end{proof}

\section{Section 45: Compactness in Metric Spaces}
\exercise{45.3}
\begin{thm}[Arzela's Theorem]
  Let $X$ be compact; let $f_n \in \mathcal{C}(X, \R^k)$. If the
  collection $\{f_n\}$ is pointwise bounded and equicontinuous, then
  the sequence $f_n$ has a uniformly convergent subsequence.
\end{thm}
\begin{proof}
  We begin with the following
  \begin{lem}[Exercise 31.4]
    Every compact metrizable space is separable.
  \end{lem}
  \begin{proof}
    For each integer $n$ consider a covering of the compact space $X$
    by a ball of radius $1/n$ centered at each point of $X$. Because
    $X$ is compact, we can extract a finite subcover. If we take the
    union of all such subcovers we are left with a countable set of
    balls that we claim is a countable base for $X$. Indeed, given an
    open set $U$ in $X$ that contains $x$ there is a $\delta > 0$ such
    that $B_\delta(x) \subset U$. If we choose $n$ such that $1/n <
    \delta/2$ then we can find some $i$ such that $x \in B_{1/n}(x_i)
    \subset B_\delta(x)$ because we chose $1/n < \delta/2$ we have
    that $B_{1/n}(x_i)$ is an element that contains $x$ and is
    contained in $U$. So we have found our countable base.

    Constructing the countable dense subset from this point is clear,
    we simply choose one point from each of the balls in out countable
    base and call this set $E$. I claim that $E$ is our countable
    dense subset. $E$ is clearly countable, so we need only verify
    that it is dense. We will show that $\overline{E}^c =
    \emptyset$. We have that $\overline{E}^c$ is open because its
    complement is closed. As a result, we use our base of balls to
    find some base element that is contained in $\overline{E}^c$, say
    $B_\delta(x_i)$. This implies that $x_n \in \overline{E}^c$. But
    this is impossible because $x_n \in E$ means that $x_n \in
    \overline{E}$. This contradiction shows that $\overline{E}^c$ is
    empty and so $E$ is a countable dense subset of $X$.
  \end{proof}
  We can now apply the lemma to the space $X$ to get a countable dense
  subset $E = \{x_1,x_2,\ldots\}$ in $X$. Our next goal is to capture
  the convergence of a subsequence of the $f_n$ by keeping track of
  which indices correspond to the sequence of convergent
  functions. More precisely, we have the following
  \begin{claim}
    There exists a decreasing chain of infinite sets $C_0 \supset C_1
    \supset C_2 \supset \cdots$ with the property that
    $\lim_{n\to\infty} f_n(x_j)$ exists for $j \leq k$ if $n \to
    \infty$ within $S_k$.
  \end{claim}
  \begin{proof}
    The idea is that each of the $C_k$ captures the indices of a
    subsequence of the $f_n$ that converges. With this in mind, we set
    $C_0 = \N$ and consider the set of points $\{f_n(x_k) \suchthat n
    \in C_0\}$ is a bounded sequence in $\R^k$ and therefore has a
    convergent subsequence. Let $C_1$ be the indices of the elements
    of the convergent subsequence. then $C_1 \subset C_0$ and $C_1$ is
    infinite. We continue inductively defining $C_k$ by considering
    the set $\{f_n(x_k) \suchthat n \in C_{k-1}\}$. As before, this is
    a bounded sequence in $\R^k$ and has a convergent subsequence and
    we set $C_k$ to be the set of indicees of that convergent
    subsequence. So we have that we have that $f_n(x_k)$ exists as
    $n\to\infty$ within $C_k$. This completes the construction.
  \end{proof}
  Let $i_k$ be the $k^{\text{th}}$ term of $C_k$ in the preceeding
  lemma and set
  \[
  C = \{i_1, i_2, \ldots\}
  \]
  By construction, there are at most $k-1$ terms of $C$ that are not
  in $C_k$ and so $\lim_{n\to\infty} f_n(x)$ exists for every $x\in E$
  as $n\to\infty$ within $C$.

  Now choose an $\epsilon > 0$ and apply the equicontinuity of the
  $f_n$ to find a $\delta > 0$ such that $d(p,q) < \delta$ implies
  that $|f_n(p) - f_n(q)| < \epsilon/3$ for all $n$. Cover $X$ with a
  set of balls of radius $\delta/2$ centered at each point. Because
  $X$ is compact we can find a finite subcover $B_1, B_2,\ldots, B_M$
  of open balls. Because $E$ is dense in $X$ there are points $p_i \in
  B_i \cap E$ for $i \leq M$. Moreover, we see that $p_i \in E$ implies
  that $\lim_{n\to\infty} f_n(p_i)$ exists within $C$. Hence, we can
  find an integer $N$ large enough that $|f_m(p_i) - f_n(p_i)| <
  \epsilon/3$ for $i \leq M$ and $n,m > N$ and $n,m \in C$.

  Now we just use the triangle equality to get the uniform
  convergence. Pick an $x\in X$. then $x \in B_i$ for some $i$ and
  $d(x,p_i) < \delta$. If we choose $\delta$ and $N$ as above we have that
  \begin{align*}
    |f_m(x) - f_n(x)| &\leq |f_m(x) - f_m(p_i)| + |f_m(p_i) -
    f_n(p_i)| + |f_n(p_i) - f_m(x)| \\
    &\leq \epsilon/3 + \epsilon/3 + \epsilon/3 \\
    &\leq \epsilon
  \end{align*}
  whenever $m,n > N$ and $m,n \in S$. This is precisely what it means
  for this subsequence to converge uniformly.
\end{proof}

\exercise{43.5}
\begin{thm}
  Let $X$ be a locally compact Hausdorff space and let
  $\mathcal{C}_0(X,\R)$ be the space of continuous functions on $X$
  that vanish at infinity with the uniform topology. A subset
  $\mathcal{F}$ of $\mathcal{C}_0(X,\R)$ has compact closure if and
  only if it is pointwise bounded, equicontinuous, and vanishes
  uniformly at infinity.
\end{thm}
\begin{proof}
  Let $X$ be a locally compact Hausdorff space. Then $X$ admits a
  one-point compactification $Y = X \cup \{\Omega\}$. We then define a map
  \begin{align*}
  F: \mathcal{C}_0(X,\R) &\to \mathcal{C}(Y,\R) \\
  f &\mapsto \bar{f}
  \end{align*}
  Where $\bar{f}$ is defined by
  \[
  \bar{f}(x) =
  \begin{cases}
    f(x) & x \in X \\
    0 & x = \Omega
  \end{cases}
  \]
  Then we first establish the following
  \begin{claim}
    The function $\bar{f}$ as defined above is continuous under the
    hypotheses of the theorem.
  \end{claim}
  \begin{proof}
    It is clear that $\bar{f}$ is continuous on $X$ because $\bar{f}$
    restricts to $f$ on $X$ and $f$ is continuous. So we only need to
    check that $\bar{f}$ is continuous at $\Omega$. Let $U$ be a
    neighborhood of $0$ in $\R$, then we have that $g^{-1}(U)$
    contains $\Omega$. We can find some interval $(-\delta, \delta)
    \subset U$, and use the fact that $\bar{f}$ vanishes at infinity
    to get some compact $K$ in $X$ such that $f(X - K) \subset
    (-\delta, \delta)$ and so $Y-K \subset \bar{f}^{-1}(U)$ and $Y-K$
    is a neighborhood of $\Omega$ in $Y$. Hence, $\bar{f}$ is
    continuous at $\Omega$.
  \end{proof}
  Now we introduce the metric structure on $\mathcal{C}_0(X,\R)$ and
  $\mathcal{C}(Y,\R)$. Let $d_0,d$ be the metrics induced by the
  sup-norm on $\mathcal{C}_0(X,\R)$ and $\mathcal{C}(Y, \R)$. Note
  that these metrics are well-defined because continuous functions on
  $X$ are necessarily bounded. We want to show that $E$ is an isometry
  of $\mathcal{C}_0(X,\R)$ with a closed subspace of
  $\mathcal{C}(Y,\R)$. Indeed, pick $f,g \in \mathcal{C}_0(X,\R)$ and
  observe that
  \begin{align*}
    d_0(f,g) &= \sup_{x\in X} |f(x) - g(x)| \\
    &= \sup_{x \in Y}|\bar{f}(x) - \bar{g}(x)| \\
    &= d(\bar{f},\bar{g})
  \end{align*}
  Note that equality holds moving from the first line above to the
  second because the point at infinity only adds $0$ to a set of
  non-negative numbers, and therefore will not be the supremum unless
  $f$ is identically zero on $X$ and equality holds nonetheless. $E$
  is also continuous because we simply take $\epsilon =
  \delta$. Hence, $E$ is a continuous, bijective and distance
  preserving map, and therefore an isometry (and, consequently, a
  homeomorphism).

  Next, we need to find how $E(\mathcal{C}_0(X,\R))$ sits inside of
  $\mathcal{C}(Y,\R)$. Because each of the $f \in \mathcal{C}_0(X,\R)$
  vanishes at infinity, the corresponding $\bar{f} \in
  \mathcal{C}(X,\R)$ must satisfy $g(\Omega) = 0$. This follows from
  the fact that $\bar{f}$ is continuous and bounded above by
  $\epsilon$ outside of an arbitrarily large compact set. Because we
  can find such a set for each $\epsilon$, we can let $\epsilon \to 0$
  and get that $f(x) \to 0$ as $x \to \Omega$. Then continuity
  guarantees that we can exchange limiting operation and function
  application so
  \[
  \lim_{x \to \Omega} f(x) = f(\lim x \to \Omega) = 0
  \]

  Now we will see that $E(\mathcal{C}_0(X,\R))$ is closed. Well, we
  recall that each of the functions in $E(\mathcal{C}_0(X,\R))$ are
  equal to 0 when evaluated at $\Omega$. Hence, if we consider the
  evaluation map given by
  \begin{align*}
    A: Y \times \mathcal{C}(Y, \R) &\to \R \\
    x \times f \to f(x)
  \end{align*}
  Well $E(\mathcal{C}_0(X,\R)) = A^{-1}(0)$, which is the inverse
  image of a closed set (singletons are closed in $\R$) under a
  continuous function, and is thus closed.

  Now we are at a point where we can apply the Ascoli theorem. We have
  seen that $E(\overline{\mathcal{F}})$ is a homemorphism. Then we
  apply Ascoli's theorem to $E(\mathcal{F})$ and get that it has
  compact closure if and only if it is equicontinuous and pointwise
  bounded. Above, we saw that this happens if and only if
  $\mathcal{F}$ is pointwise bounded, equicontinuous, and vanishes
  uniformly at infinity. We then apply the fact that $E$ is a
  isometric homeomorphism so that $E(\overline{\mathcal{F}}) =
  \overline{E(\mathcal{F})}$. And so $\mathcal{F}$ as compact closure
  if and only if it is equicontinuous, pointwise bounded, and vanishes
  uniformly at infinity.
\end{proof}

\section{Section 46: Pointwise and Compact Convergence}
\exercise{46.7}

 We begin with a map
\begin{align*}
  \Phi: \mathcal{C}(X,Y) &\times \mathcal{C}(Y,Z) \lra \mathcal{C}(X,Z) \\
  f &\times g \lra g\circ f
\end{align*}
which is the composition operator on the two function spaces. To show
that $\Phi$ is continuous, we will prove that the inverse image of
each open sub-basis element is open. Indeed, consider the set open
$(K,U)$ where $K$ is compact in $X$ and $U$ is open in $Z$. Suppose
that $(f,g) \in \Phi^{-1}(K,U)$ so that (following the hint) $f(K)
\subset g^{-1}(U)$. Because $f(K)$ is compact, $g^{-1}(U)$ is open in
$Y$, and $X$ is locally compact and regular we can (APPLY A THEOREM
TO) see that there is some compact subset $L$ such that $f(K) \subset
\text{int}(L)$ and $L \subset g^{-1}(U)$. The set $(K,\text{int}(L))
\times (L, U)$ is an open set in $\mathcal{C}(X,Y) \times
\mathcal{C}(Y,Z)$. Furthermore, it contains the point $(f,g)$.

So if we pick any $h \in (K, \text{int}(L)$ and $i \in (L, U)$ then we have
\[
(i\circ h)(K) \subset i(\text{int}(K)) \subset i(L) \subset U
\]
This implies that $(h,i) \in (K,U)$. We have shown that
\[
(K,\text{int}(L)) \times (L, U) \subset \Phi^{-1}(K,U)
\]
So $\Phi^{-1}(K,U)$ is open, and so $\Phi$ is continuous.

\section{Section 48: Baire Spaces}
\exercise{48.7}
\begin{thm}
  If $D$ is a countable dense subset of $\R$, there is no function $f:
  \R \to \R$ that is continuous precisely at the points of $D$.
\end{thm}
\noindent{\textit{Proof.}}
\begin{enumerate}
\item [\textbf{(a)}] Following the hint for the problem we begin by
  noting the following
  \begin{lem}
    Given a function $f: \R \to \R$ we have the following
    \begin{enumerate}
    \item[\textbf{(i)}] $f$ is continuous if and only if
      \[
      D_f(x)\lim_{\substack{x\in U\\ \diam{U} \to 0}} \diam{f(U)} = 0
      \]
    \item[\textbf{(ii)}] The set
      \[
      E_n = \{x \in \R \suchthat D_f(x) < 1/n\}
      \]
      is open.
    \end{enumerate}
  \end{lem}
  \begin{proof}
    For the first claim, this is just the definition of continuity in
    $\R$ phrased in the language of diameters. If $f$ is continuous
    then we have for any $\epsilon > 0$ we can find $\delta > 0$ such
    that $|x-y| < \delta$ implies $|f(x) - f(y)| < \epsilon$. If you
    give an epsilon and the limit holds, then we simply choose the
    first $U$ such that the diameter is less than $\epsilon$ and set
    $\delta = \diam(U)$. Conversely, if $f$ is continuous, then we
    simply say that for each $U$ containing $x$ we have that $\diam(U)
    \to 0$ implies that $|x-y| \to 0$ for $y \in U$. This immediately
    gives the result.

    For the second part, note that if $x \in E_n$ then we can find
    some $U$ such that $\diam{U}< \delta$ so that $\diam{f(U)} <
    1/n$. So if $y \in B_{\delta/2}(x)$ then $y \in E_n$ because
    \[
    \sup_{p,q \in B_{\delta/2}(y)}|f(p) - f(q)| \leq \sup_{p,q in
      B_{\delta(x)}} |f(p_ - f(q)| < 1/n
    \]
    So $B_{\delta/2}(x)$ is an open ball containing $x$ that is wholly
    contained in $E_n$, hence $E_n$ is open.
  \end{proof}
  This lemma shows that we can write
  \[
  C = \bigcap_{n=1}^\infty E_{n}
  \]
  where $C$ are the points of continuity of $f$.The first part of the
  lemma says the points int the intersection are precisely the
  continuity points of $f$ and the second part of the lemma verifies
  that this set is a $G_\delta$, as desired.
\item [\textbf{(b)}] The result is an immediate consequence of the following
  \begin{lem}
    A dense $G_\delta$ set in $\R$ is generic.
  \end{lem}
  \begin{proof}
    Suppose that $Y$ is a dense $G_\delta$ set. Then we can write $Y =
    \bigcap_{n=1}^\infty U_n$, where each of the $U_n$ is open and
    dense (each must be dense because otherwise the intersection could
    not be). Suppose to the contrary that $Y$ is of the first category
    and we could also write $Y = \bigcup_{n=1}^\infty$, where each one
    of the $W_n$ is nowhere dense. We then note that each of the
    $U_n^c$ is nowhere dense and rewrite
    \[
    \R = Y \cup Y^c = \bigcup_{n=1}^\infty W_n \cup \bigcup_{n=1}^\infty U_n^c
    \]
    Then $\R$ is the countable union of nowhere dense sets, and
    therefore is first category in itself. which contradicts the Baire
    Category theorem.
  \end{proof}
  Applying this lemma gives immediately that if $D$ is a countable
  dense set, then $D$ cannot be $G_\delta$, and therefore cannot be
  the set of continuity points of any function because by the previous
  part, those must be $G_\delta$.
\end{enumerate}
\hfill\qedsymbol

\exercise{48.8}
\begin{thm}[Uniform Boundedness Principle]
  Let $X$ be a complete metric space and let $\mathcal{F}$ be a subset
  of $\mathcal{C}(X,\R)$ such that for each $a \in X$ the set
  \[
  \mathcal{F}_a = \{f(a) \mathrel{|} f \in \mathcal{F}\}
  \]
  is bounded. Then there is a nonempty open set $U$ of $X$ on which
  the functions in $\mathcal{F}$ are uniformly bounded, that is, there
  is a number $M$ such that $|f(x)| \leq M$ for all $x \in U$ and $f
  \in \mathcal{F}$.
\end{thm}
\begin{proof}
  We will begin by proving the result for sets $Y \subseteq X$ of the
  second category and the theorem will follow because $X$ is a compact
  metric space, and therefore a Baire space.

  Following the hint, we begin by assigning to each integer $N$ the set
  \[
  A_n = \{x\suchthat \forall f \in \mathcal{F}, |f(x)| \leq N\}
  \]
  Then, the fact that each of the $\mathcal{F}_a$ is bounded allows us
  to write
  \[
  A = \bigcup_{N = 1}^\infty A_N
  \]
  We then note that each the $A_N$ must be closed because we can write
  \[
  A_N = \bigcap_{f \in \mathcal{F}} A_{N,f}
  \]
  And each of the $A_{N,f} = \{f\suchthat |f(x)| \leq N\}$ must be
  closed because $f$ is assumed continuous.

  Because $A$ is second category there is $A_N$ for some sufficiently
  large $N$ that must have a non-empty interior. So we can find an
  $x_0 \in X$ and an $r > 0$ such that the ball $B_r(x_0)$ is
  contained in $A_N$. For each of the $f \in \mathcal{F}$, we get $x
  \in B_r(x_0)$ implies that $|f(x)| \leq N$. This is the open ball
  that we were looking for by construction.
\end{proof}

\section{Section 55: Retractions and Fixed Points}

\exercise{55.3}

 Consider the intersection of the sphere with the first
octant (all non-negative coordinates) in $\R^3$. Graphically, this is
one ``slice'' of $S^2$ in 3-space. We know how to parameterize such a
space via polar coordinates, as $(r,\theta, \phi)$. All we do need to
do is make sure that the map from our slice to the whole sphere is a
homeomorphism. this is clear if we simply use the map $(r,\theta,
\phi) \mapsto (r, 4\theta, 4\phi)$. Each of these component maps is
linear and hence continuous and bijective, so this is a homeomorphism. 

Because $A$ has no singularities and each of the components of $B$ are
non-negative. We can then map the $F: B \to B$. Because $Ax$ is all
non-negative We have that this map is well defined as $F(x) =
Ax/\norm{Ax}$. Because $B$ is homeomorphic to itself we can apply
Brower's fixed point theoerm and then see immediatly that
\[
Ax_0/\norm{Ax_o} \text{ implies } Ax_0 = \norm{Ax_0}x_0
\]
So again, this is the same as
\[
Ax_0 = \norm{Ax_0}x_0
\]
The right hand side is a non-negative constant and the formula
immediately gives that $x_0$ is indeed an eigenvector.

\section{Section 57: The Borsuk-Ulam Theorem}

\exercise{57.2}

Suppose that $g$ were not surjective and let $p$ be a point in the
complement of the image of $g$. Then define a map $h: S^2-\{p\} \to
\R^2$ to be a homeomorphism (following the hint). So we can define a
continuous function $f: S^2 \to \R^2$ by $f = h\circ g$. We then apply
the Borsuk-Ulam theorem to find a pair of points such that $f(x) =
f(-x)$, and because $h$ was a homeomorphism we see that this must be a
contradiction. This shows that $g$ must be surjective.

\end{document}