\documentclass{article}
\usepackage[tmargin=1in,bmargin=1in,lmargin=1.5in,rmargin=1.5in]{geometry}
\usepackage{amsfonts,amsmath,amssymb,amsthm,relsize,fancyhdr,parskip,graphicx}

\pagestyle{fancy}
\lhead{Ben Carriel}
\chead{Math 4530 Problem Set 12}
\rhead{\today}

\parskip 7.2pt
\parindent 8pt

\DeclareMathOperator{\N}{\mathbb{N}}
\DeclareMathOperator{\Z}{\mathbb{Z}}
\DeclareMathOperator{\Q}{\mathbb{Q}}
\DeclareMathOperator{\R}{\mathbb{R}}
\DeclareMathOperator{\C}{\mathbb{C}}
\DeclareMathOperator{\capchi}{\raisebox{2pt}{$\mathlarger{\mathlarger{\chi}}$}}

\DeclareMathOperator{\divides}{\mathrel{|}}
\DeclareMathOperator{\suchthat}{\mathrel{|}}

\DeclareMathOperator{\lra}{\longrightarrow}
\DeclareMathOperator{\into}{\hookrightarrow}
\DeclareMathOperator{\onto}{\twoheadrightarrow}
\DeclareMathOperator{\bijection}{\leftrightarrow}

\newcommand{\problem}[1]{\noindent{\textbf{Problem #1}}\\}
\newcommand{\problempart}[1]{\noindent{\textbf{(#1)}}}

\newcommand{\der}[2]{\frac{\partial #1}{\partial #2}}
\newcommand{\norm}[1]{\|#1\|}
\newcommand{\diam}[1]{\text{diam}(#1)}

\DeclareMathOperator{\im}{\text{im}}

\newtheorem*{thm}{\\ Theorem}
\newtheorem*{lem}{\\ Lemma}
\newtheorem*{claim}{\\ Claim}
\newtheorem*{defn}{\\ Definition}
\newtheorem*{prop}{\\ Proposition}

\begin{document}

\problem{9.59.1} Choose two copies of $S^2$, say $A$ and $B$ and let $p_1
\in A$ and $p_2 \in B$ be the points such that if we let $\sim$ be
the relation that identifies $p_1$ with $p_2$ then we have that $(A\cup B)/
\sim\text{ } \approx X$. Because $S^2$ is a manifold we can find
neighborhoods $U_1$ of $p_1$ and $U_2$ of $p_2$ in $A$ and $B$ respectively
that is homeomorphic to a disk in $\R^2$. We then define $V_A = A \cup U_2$
and $V_B = B \cup U_1$. Then $V_1,V_2$ are open and their union is $X$.
Furthermore, their intersection is $U_1 \cup U_2$, which is non-empty and
path connected. We want to show that $V_A$ is simply connected. We see that
$U_2$ deformation retracts to the point $p_1$ because we can take a
homeomorphism of the straight line homotopy. Hence, $U$ has the same homotopy
type as $A$, and is therefore simply connected. The same argument shows that
$B$ is simply connected. This means that the fundamental group of $X$ is the
trivial group.

\problem{9.59.2} The problem with the proof is that it may be impossible to
pick a point not in the image of $f$. It is possible to have continuous
surjections $f: I \to S^2$. An example of such a function is the space-filling
curve $s: I \to I^2$ composed with a homeomorphism $t: I^2 \to S^2f$ so that
$t\circ s$ is a continuous surjection $I \to S^2$.

\problem{9.59.3}
\problempart{a} If we have that $h: \R^1 \to \R^n$ is a homeomorphism then $h$
has a restriction to a homeomorphism $h^*: \R^1 - \{0\} \to \R^n - \{h(0)\}$.
But, $\R^1 - \{0\}$ is disconnected and $\R^{n} - \{x\}$ is connected for any
$x \in \R^n$ and $n > 1$.

\problempart{b} If $h: \R^2 \to \R^n$ was a homeomorphism, then $h$ has a
restriction to a homeomorphism $h^*: \R^2 - \{0\} \to \R^n - \{h(0)\}$. We then
note that $\R^2 - \{0\}$ is homotopy equivalent to the circle $S^1$ and
therefore is simply connected. However, $\R^n - \{x\}$ is homotopy
equivalent to $S^{n-1}$ and therefore is simply connected when $n > 2$.
Hence, these two spaces cannot be homeomorphic.

\problem{9.59.4}
\problempart{a} If we are given that $j_*$ is the trivial homomorphism then
the fundamental group of $X$ must be generated by the image of $i_*$ alone
because it is generated by the identity and the image of $i_*$, which is the
same as just the image of $i_*$ because that image is a subgroup and
therefore contains the identity.

If $i_*$ and $j_*$ are both trivial homomorphisms then the fundamental group
of $X$ must be trivial as well because it is the subgroup generated by the
identity, which is just the identity again.

\problempart{b} Again we take two copies of $S^2$, call them $A$ and $B$.
Then choose four distinct points $p_1,p_2,p_3,p_4$ and define
\[
A' = A - \{p_1,p_2\} \text{ and } B' = B - \{p_3, p_4\}
\]
Then both $A'$ and $B'$ contain a copy of $S^1$ as a deformation retract, and
consequently have the fundamental group $\Z$. However, we note that
if $X = A' \cup B'$, then $X \cong S^2$ which has the trivial fundamental
group. So each of the $i_*$ and $j_*$ must send every loop class to the
identity and are therefore trivial homomorphisms.

\problem{9.60.5} Let $E$ be the covering space $(A_0 \cup A_1 \cup B_0 \cup
B_1$ as denoted in the diagram on pg. 375. Let $e_1$ be the point where $A_1$
intersects $B_0$ and let $e_2$ be the point where $B_1$ intersects $A_0$. Let
$\bar{\alpha}$ be a path in $A_1$ from $e_0$ to $e_1$ and $\bar{\beta}$ be a
path in $B_1$ from $e_0$ to $e_2$. Then we set $\alpha = p \circ
\bar{\alpha}$ and $\beta = p \circ \bar{\beta}$, which are loops in $X$ at
$x_0$. Let $\alpha_\ell$ be the lift of $\alpha$ at $e_2$ and let $\beta_\ell$
be the lift of $\beta$ at $e_2$. The image of $\alpha_\ell$ is contained in
$A_0 \cup A_1$ and therefore in the component of $A_0$ of the set that
contains its initial point. Thus, $\alpha_\ell$ is a loop at $e_2$.
Analogously, we have that $\beta_\ell$ is a loop at $e_1$. We now compute the
lift of $f * g$ at $e_0$ to be $\bar{\alpha} * \beta_\ell$ and similarly that
of $g * f$ is $\bar{\beta} * \alpha_\ell$. So if we use the lifting
correspondence
\[
\phi: \pi_1(X,x_0) \to p^{-1}(x_0)
\]
which is determined by
\[
\phi([f*g]) = e_1 \text{ and } \phi([g*f]) = e_2
\]
we see that $[f]*[g] \neq [g] * [f]$ and so $\pi_1(X,x_0$ is not abelian.

\problem{13.79.1} We have that $S^n$ is simply connected for $n > 1$.
Consequently, we can apply Lemma 9.79.1 to see that any continuous map
$f: S^n \to S^1$ lifts to a map $\tilde{f}: S^n \to \R$. The map is clearly
nullhomotopic by a straight-line homotopy and so $f = p \circ \tilde{f}$ is
nullhomotopic as well.

\problem{13.79.2}
\problempart{a}Let $f: P^2 \to S^1$ be a continuous map. Consider the induced
map $f_*$ on the fundamental groups. We know that
\begin{align*}
  \pi_1(P^2, x) &\cong \Z_2 \\
  \pi_1(S^1, y) &\cong \Z
\end{align*}
From algebra we know that the only homomorphism $\varphi: \Z_2 \to \Z$ is the
trivial homomorphism. Hence, we have that $f_*(\pi_1(P^2, x)) = 0$. So we can
lift $f$ to a map $\tilde{f}: P^2 \to \R$ as in the previous problem. As a
result, we again get that $f$ is nullhomotopic.

\problem{13.79.3}

\problem{13.81.2}
\problempart{a} Clearly, this is a 2-sheeted cover from the definition. Hence
, we have that $H_0 = p_*(\pi_1(E,e_0))$ has index 2 in $\pi_1(X,x_0)$. Then
we recall that any subgroup of index 2 is normal and so $H_0 \unlhd \pi_1(X)$
.

\problempart{b} We follow the argument given in example 13.81.2. So if $p:
(E, e_0) \to (B_, b_0)$ is a covering map and $h$ is a covering
transformations, then any loop at $b_0$ lifts to a loop at $e_0$ iff it lifts
to a loop at $h(e_0)$. But we can see that a loop that wraps $B$ once lifts
to a loop at $e_1$ but not a loop at $e_0$. Because $h(e_0)$ is one of $e_0,
e_1,e_2$ we see that $h(e_0)$ is not $e_1$. Analogously, we see that $h(e_0)$
is not equal to $e_2$ by the same argument for loops wrapping around $A$.
So we have that $h(e_0) = e_0$ and so $h$ is the identity and therefore the
covering is not regular. 

\problempart{c}

\problempart{d}

\end{document}