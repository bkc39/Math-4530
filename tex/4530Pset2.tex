\documentclass{article}
\usepackage{amsfonts,amsmath,amssymb,amsthm,fancyhdr,parskip,graphicx}

\pagestyle{fancy}
\lhead{Ben Carriel}
\chead{Math 4530 Problem Set 2}
\rhead{\today}

\parskip 7.2pt
\parindent 8pt

\DeclareMathOperator{\Z}{\mathbb{Z}}
\DeclareMathOperator{\Q}{\mathbb{Q}}
\DeclareMathOperator{\R}{\mathbb{R}}

\DeclareMathOperator{\divides}{\mathrel{|}}
\DeclareMathOperator{\suchthat}{\mathrel{|}}

\DeclareMathOperator{\interior}{\text{Int}}
\DeclareMathOperator{\bd}{\text{Bd}}

\DeclareMathOperator{\lra}{\longrightarrow}
\DeclareMathOperator{\into}{\hookrightarrow}
\DeclareMathOperator{\onto}{\twoheadrightarrow}
\DeclareMathOperator{\bijection}{\leftrightarrow}
\DeclareMathOperator{\hookta}{\lhook\joinrel\xrightarrow{\ \ \ \ }}

\newcommand{\problem}[1]{\noindent{\textbf{Problem #1}}\\}
\newcommand{\problempart}[1]{\noindent{\textbf{(#1)}}}

\newcommand{\der}[2]{\frac{\partial #1}{\partial #2}}

\newtheorem*{thm}{Theorem}
\newtheorem*{lem}{Lemma}
\newtheorem*{claim}{Claim}
\newtheorem*{defn}{Definition}
\newtheorem*{prop}{Proposition}

\begin{document}

\problem{17.6}
\problempart{a} Suppose that $A \subset B$. Then we must have that any closed set $C$ containing $B$ must also contain $A$ because $A \subset B \subset C$. Hence, $A \subset \overline{B}$. So $\overline{B}$ is a closed set containing $A$ and so by definition $\overline{A} \subseteq \overline{B}$ because $\overline{A}$ is the smallest closed set containing $A$. \\

\problempart{b}  The set $\overline{A \cup B}$ is the smallest closed set containing both $A$ and $B$. With this in mind we see that $\overline{A \cup B} \subseteq \overline{A} \cup \overline{B}$ because $A \subset \overline{A} \cup \overline{B}$ and $B \subset \overline{A} \cup \overline{B}$ and $\overline{A} \cup \overline{B}$ is closed because it is the union of closed sets. to see the reverse containment, choose any $x \in \overline{A} \cup \overline{B}$. Theorem 17.5 for any open set $U$ containing $x$ we have that either $ U \cap (A) \neq \emptyset$ or $U \cap B \neq \emptyset$ because $x \in A$ or $x \in B$. So
\[
(U \cap A) \cup (U\cap B) = U \cap (A \cup B) \neq \emptyset
\]
But $U \cap (A \cup B) \neq \emptyset$ implies that $x \in \overline{A \cup B}$. This proves the reverse equality and we are done. \\

\problempart{c} We can follow the outline of the last proof to see that for any $x \in \bigcup_\alpha A_\alpha$ and any open set $U$ which contains $x$ that there is some $\alpha$ such that $U \cap A_\alpha \neq \emptyset$. So this means that 
\[
\bigcup_\alpha (U \cap A_\alpha) = U \cap \bigcup_\alpha A_\alpha \neq \emptyset  
\]
Note that $ U \cap \left(\bigcup_\alpha A_\alpha\right) \neq \emptyset$ implies that $x \in \overline{\bigcup_\alpha A_\alpha}$. To see why the other inclusion fails in this case, note that in {\bf (b)} we had a finite union. So you can take $A_n = \{1/n\}$ and set $A = \bigcup_{n=1}^\infty A_n$ then we have that 
\[
A = \bigcup_{n=1}^\infty \overline{A}_n \subset \bigcup_{n=1}^\infty \overline{A}_n \cup \{0\} = \overline{A}
\]

\problem{17.14} Let $x_n = 1/n$ be a sequence lying in $\R$ with the finite complement topology. If we take any $x \in \R$ and any open set $U$ containing $x$, we can see that this means that $|U^c| < \infty$. that means that there are only finitely many points in $\R$ that are not contained in $U$. since each $x_n \in \R$ this means that only finitely many of the points of the sequence are not contained in $U$. This means that $x_n \to x$ for each $x \in \R$.  

\problem{17.16} 
\problempart{a} We want to find the closure of the set $K = \{1/n \suchthat n \in \Z_+\}$. We have
\begin{enumerate}
\item In the standard topology it is well known that the closure of this set is $\overline{K} = K \cup \{0\}$. To see this take any convergent sequence in $K$, and it will be a subsequence of $x_n = 1/n$. So we only need to find the limit of $x_n$ as $n\to \infty$. We can see that $0$ works because for any $\epsilon > 0$ there is an $n$ such that $1/n < \epsilon$ and $x_n \to 0$. This gives that $\{0\}$ is the set of limit points of $K$. 
\item In the topology $\R_K$ the sequence the closure $\overline{K} = \emptyset$. We can see this because if we take any open set in $\R_K$, then it can be written as a union of sets of the form $(a,b) - K$, but then no open set contains any points of $K$. Hence, there are no sequences in $K$ that converge and so the closure of $K$ is empty. 
\item We showed above that in the finite complement topology the set $\overline{K} = \R$ because if we take any open set in $\R$ with this topology we must have that only finitely many of the points are not in the open set. Hence, any point in $\R$ is a limit point of a sequence in $K$. 
\item In the upper limit topology we have that a sequence converges if it ``approaches from the left''. More precisely if we have $x_n \in K$ then for any $\epsilon > 0$ there is an $N$ such that $n > N$ implies that
$L-\epsilon < x_n \leq L$ then $x_n \to L$. Then the set of limit points for the sequence $x_n = 1/n$ is empty because $x_n$ does not approach 0 from the left. 
\item We now endow $\R$ with the topology generated by the basis 
\[
\mathcal{B} = \{(-\infty, a) \suchthat a \in \R\} 
\]
Here we will see that the set of limit points is $\R_+$. To see this, choose any $x > 0$. Observe that $x \in (-\infty, y)$ for each $y \geq x$. If $x \geq 0$ then we can choose any $\epsilon > 0$ and eventually $1/n < \epsilon$ and so there can only be finitely many terms in $K$ that are not in $(-\infty, y)$ for $y > x$.  
\end{enumerate}

\problempart{b} Now we will see which of the sets is Hausdorff.
\begin{enumerate}
\item $\R$ with the standard topology is a Hausdorff space. To see this choose any distinct $x,y \in \R$. We can then set $\ell = d(x,y) / 2$. If $x \neq y$ then this quantity will be positive. We then choose the open ball centered at $x$ and $y$ respectively of radius $\ell$ and note that these are two disjoint neighborhoods. Then choose any neighborhood of $x$ with diameter $r$. Then there is a neighborhood of $x$ with diameter $r/2$. Finally, take $y \in U$, an open neighborhood of $x$. Then if $U$ has diameter $r$ and $d(x,y) = \ell$ then we can see that the open ball of radius $(r - \ell)/2 $ will lie in $U$ and contains $y$. So $\R$ is Hausdorff.
\item The topological space $R_K$is also Hausdorff. This follows from the fact that the $K-$topology is strictly finer than the standard topology. Thus, the open sets mentioned above will also work in $\R_k$ because they are still open and disjoint.
\item The finite complement topology is not Hausdorff. To see this choose to points $x,y \in \R$. Then any neighborhood of these points say $U_x$ and $U_y$, respectively, must contain all but finitely many points. However, $U_x\cap U_y \neq \emptyset$ because they could only possibly differ in a finite number of points, and they are both infinite, meaning that they share at least one point. 
\item We will proceed as above, first showing that the upper limit topology is strictly finer than the standard topology, and hence $\R_u$ is Hausdorff. Let $\mathcal{T}$ be the standard topology on $\R$ and $\mathcal{T}'$ the upper limit topology. Given a basis element $(a,b) \in \mathcal{T}$ and a point $x \in (a,b)$ the basis element $(x,b] \in \mathcal{T}'$ is contained in $(a,b)$. However, given an element $[x, b)$ in $\mathcal{T}'$ there is no open interval containing $x$ that lies in $[x,b)$. Hence $\mathcal{T}'$ is strictly finer than $\mathcal{T}$.   
\item The topology generated by the basis 
\[
\mathcal{B} = \{(-\infty, a) \suchthat a \in \R\} 
\]
is Hausdorff. We can verify this fact by choosing two points $x,y \in \R$ with this topology and without loss of generality suppose that $x < y$. Then any open set containing $x$ must also contain an element $(-\infty, a)$ with $x < a$. But then $x < (x+y)/2 < y$ and if we set $a = (x+y)/2$ then $x \in (-\infty, a)$ but $y \not\in (-\infty, a)$ and the two sets are disjoint. Hence, $\R$ is not Hausdorff in this topology.
\end{enumerate}
Now we will see which of these spaces satisfy the $T_1$ axiom.
\begin{enumerate}
\item In the standard topology we have that $\R$ is $T_1$. This follows from the third axiom of being a Hausdorff space.
\item Similarly, $\R_K$ is Hausdorff and, consequently, $T_1$.
\item In this case we have that $\R$ is not Hausdorff, but is $T_1$. To do this, take the open set $X\setminus\{y\}$, an open set containing $x$ but not $y$, and $X\setminus\{x\}$, an open set containing $y$ but not $x$. This verifies that $\R$ is $T_1$ in the finite complement topology. 
\item Again, $\R$ is a Hausdorff space in the upper limit topology and hence it must also be $T_1$.
\item In this case we have that $\R$ is $T_1$ because it is Hausdorff. 
\end{enumerate}

\problem{17.19}
\problempart{a} First we will see that $\interior A \cap \bd A = \emptyset$. Suppose that $x \in \bd A$. Then we must have that $x \in \overline{A}$ and $x \in \overline{(X - A)}$. This means that any neighborhood $U$ of $x$ intersects both $A$ and $X-A$. However, for any $y \in \interior A$ there is an open neighborhood of $y$ that is wholly contained in $A$, and therefore cannot intersect $X-A$. Thus, we have that $\interior A \cap \bd A = \emptyset$. Now we will show that $\overline{A} = \interior A \cup \bd A$. First observe that $\interior A \subset \overline{A}$ because $\interior A \subset A \subset \overline{A}$. Furthermore, $\bd A \subset \overline{A}$ by the definition, and so $\interior A \cup \bd A \subseteq \overline{A}$. In the reverse direction take any $x \in \overline{A}$. If $x \in \interior A$ then we are done. If not, then we must have that $x \in \bd A$ because $x \not\in \interior A$ means that there is no open set in $A$ which contains $x$. Then $x \in \overline{(X-\interior A)} \supset \overline{(X-A)}$. this means that $x \in \bd A$ and we are done.  \\
\problempart{b} Now we need to show that $\bd A = \emptyset$ if and only if $A$ is both open and closed.   We proceed as follows, $\bd A = \emptyset$ means by definition that $\overline{A} \cap \overline{(X-A)} = \emptyset$. Hence, the set of $x$ that satisfy such a constraint are those for which $x \in U$ open in $X$ such that $U\cap A = \emptyset$ and $U\cap X-A = \emptyset$ meaning that $U$ is an open neighborhood of $x$ fully contained in both $A$ and $X-A$. This implies that $A, X-A \in \mathcal{T}$. That is, $A$ is both closed and open. 
\problempart{c} Recall that if $U$ is open then $U = \interior U$. Furthermore, by part {\bf (a)} we have that $\overline{U} = \interior U \cup \bd U$. But then 
\[
\overline{U} = U \cup \bd U
\]
The union is disjoint and so $\overline{U} - U = \bd U$ as desired. \\
\problempart{d} This is not true. Consider $U = \R - \{x\}$ for any $x \in \R$. Then we have that $U$ is open and $ U \subset \interior\overline{(\R - \{x\})} = \R$. This containment is clearly strict so $U \subset \interior \overline{U}$. 

\problem{18.2} Suppose that we have the constant function defined by $x \mapsto y$ for some given $y$ and every $x$. Then $y = f(x)$ for each $x$, but $y$ is not a limit point of $f(A)$ because it contains no points besides itself. Therefore, every neighborhood of $y$ contains just one point of $f(A)$, and as a result is not a limit point.   

\problem{18.7} 
\problempart{a} Let $U \subset \R$ be open and choose any $x \in f^{-1}(V)$. Then we can find some $y \in U$ such that $y = f(x)$. $U$ is open and so we can find some $\epsilon > 0$ such that $B_{\epsilon}(y) \subseteq U$. And then by the continuity of $f$ we can therefore find a $\delta > 0$ such that $f([x, x+\delta)) \subseteq B_\epsilon(y) \subset U$. But for sufficiently small $\delta$, $[x,x+\delta) \subset f^{-1}(U)$ and is basis element for the topology of $\R_\ell$ and so $f^{-1}(U)$ is open in $\R_\ell$. Hence, $f$ is continuous. 

\problem{18.8}
\problempart{a} Suppose that $f,g: X \to Y$ are continuous and let $U \{x \suchthat f(x) \leq g(x)\}$. To show that $U$ is closed we will show that its complement $X-U$ is open. By definition $X-U = \{x \suchthat f(x > g(x))\}$. Then we can write, 
\[
X - U = \bigcup_{y\in Y}\{f(x) > y > g(x)\}
\]
Then if $y' > y$ and $(y,y') = \emptyset$ then 
\[
\bigcup_{y,y'} \{f(x)>y \text{ and } g(x) < y'\}
\] 
which is open in $X$. Hence, $X-U$ is open and as a result $U$ is closed. \\
\problempart{b} We define $h: X \to Y$ via 
\[
h(x) = \min\{f(x), g(x)\}
\]
We can decompose $X$ into two pieces, $U = \{x \suchthat f(x) \leq g(x)\}$ and $\overline{X-U} = \{x \suchthat f(x) \geq g(x)\}$. These two sets intersect on $\{x \suchthat f(x) = g(x)\}$. Furthermore, $f,g$ are continuous when restricted to either $U$ or $X-U$ by definition. Hence, we can apply the pasting lemma to see that $h$ is continuous. \\
\problem{18.9}
\problempart{a} We are given a finite collection $A_\alpha$ such that $A_\alpha$ is closed for each $\alpha$ and $X = \bigcup_\alpha A_\alpha$. Choose any closed $V \subset Y$ and note that 
\[
f^{-1}(V) = f^{-1}(f(X) \cap V)
\]
But because $X = \bigcup_\alpha A_\alpha$, 
\[
f(X) = f\left(\bigcup_\alpha A_\alpha) = \bigcup_\alpha f(A_\alpha)\right)                      
\]
So
\[
f^{-1}(V) = f^{-1}\left(\bigcup_{\alpha}f(A_\alpha) \cap V\right) = \bigcup_\alpha f^{-1}\left(f(A_\alpha) \cap V\right)
\]
Each of the terms $f(A_\alpha) \cap V = f|_{A_\alpha}(V)$ is a finite intersection of closed sets and hence, is closed. We then use the fact that $f$ is continuous to see that their inverse image under $f$ is also closed. This means that the last union is a finite union of closed sets, and therefore closed. This means that $f$ is continuous. \\
\problempart{b} First we let our collection of closed sets $\{A_n\} = \{[1/(n+1), 1/n] \suchthat n \in \Z_+\}$. In this case $\{A_n\}$ is not finite. Next, define the function $f: \R \to \R$ by
\[
f(x) = 
\begin{cases}
0 & x\in A_n \\
1 & x\in  (-\infty, 0)
\end{cases}
\]
Then $f$ is continuous on $A_n$ for each $n$ and it is also continuous on $(-\infty, 0)$, but it is not continuous on $(-\infty, 1]$ (the discontinuity is at 0). The part of the proof that failed is the fact that $f^{-1}(V)$ is a union of closed sets, but the infinite union of closed sets need not be closed. \\
\problempart{c} We proceed as we did in {\bf (a)}. Let $V \subset Y$ be closed and suppose that $\{A_\alpha\}$ is a locally finite collection of closed sets in $X$ with $X = \bigcup_\alpha A_\alpha$. As before we note that 
\[
f^{-1}(V) = f^{-1}\left(\bigcup_\alpha (f^{-1}(A_\alpha) \cap V)\right) = f^{-1}\left(\bigcup_\alpha f|_{A_\alpha}(V)\right) 
\]
If we set $A = f^{-1}\left(\bigcup_\alpha f|_{A_\alpha}(V)\right) $, then we are left to show that $A$ is closed. We will instead show that the set $X - A$ is open. Choose any $x \in X-A$. Then there is a neighborhood $U$ of $x$ that only intersects finitely many of the $A_\alpha$. Denote this subcollection by $A_1, A_2, \ldots, A_n$. We now use the continuity of $f$ to see that the inverse image $f^{-1}|_{A_i}(V)$ is closed in $A_i$. Because $x \not\in f^{-1}(V)$ there exists some neighborhood $U_i$ such that $U_i \cap A_i = \emptyset$. We then can see that the neighborhood $U \cap U_1 \cap \cdots \cap U_n$ is a finite intersection of open sets, which does not intersect the $A_\alpha$. Hence, $X-A$ is open, and we are done.  

\problem{}
We need to establish the following \\
\begin{prop}
Let $S_\Omega$ denote the smallest well-ordered set as outlined in Munkres. Then for each $\alpha < \Omega$, $S_\alpha$ is order-isomorphic with a subset of the real numbers in the usual order. 
\end{prop}
\begin{proof}
We will construct an order isomorphism via the principle of recursive definition in $S_\Omega$, which is possible because we know that $S_\Omega$ is well ordered. Choose any section $S_\alpha$, and note that by definition such a set is countable. Let us denote is elements by $s_1,s_2, \ldots$. The main observation in this proof is the following
\begin{lem}
Any well ordered subset $X \subset \R$ is countable. 
\end{lem}
\begin{proof}
Suppose that $X$ is both uncountable and well ordered. Let $s(x)$ be the successor function on $X$. So $s(x)$ is the smallest element $> x$, which exists because we assumed $X$ was well-ordered. We now define a ``distance'' function $d$ such that $x \mapsto s(x) - x$, which must be greater than 0. We then set $X_n = \{x \in X \suchthat d(x) > 1/n\}$. Then each $X_n$ is countable because for distinct $x,y \in X$ $|x - y| > 1/n$. Next, notice that $X \subset \bigcup_{n\in \Z_+} X_n$ because $d(x) > 0$ for every $x$. But then we must have that $X \subset \bigcup_{n\in \Z} X_n$, which is countable. So $X$ must also be countable. this contradiction establishes the result.
\end{proof}
This result would lead us to believe that we should look to a countable set in $\R$ to contain the order preserving isomorphic image. Naturally, we look to subsets of $\Q$. Let $q_1,q_2,\ldots$ be an enumeration of the rationals. We define our isomorphism recursively as follows, start with $f(s_1) = q_1$. If $R$ is the relation in $S_\Omega$, then we choose $f(s_n)$ such that it preserves the order relations in $s_k$ for $k < n$. This is possible because at each step in the iteration of the definition, there are only finitely may constraints, and $\Q$ is both unbounded and dense, so we can place $f(s_n)$ in between $f(s_i)$ and $f(s_j)$, when appropriate. Furthermore, if $s_n R s_k$ for $k < n$ then we can always choose $f(s_n) < f(s_k)$ for all $k < n$ because $\Q$ is unbounded. The same statement holds for $f(s_k) < f(s_n)$. Iterating this process (countably) infinitely many times yields an order preserving isomorphism from $S_\alpha \to Q \subset \Q$ which is at most countable, and we are done. 
\end{proof}
\end{document}

