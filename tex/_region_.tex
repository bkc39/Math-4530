\message{ !name(4530final.tex)}\documentclass{article}
\usepackage[tmargin=1in,bmargin=1in,lmargin=1.5in,rmargin=1.5in]{geometry}
\usepackage{amsfonts,amsmath,amssymb,amsthm}
\usepackage{relsize,fancyhdr,parskip}
\usepackage{graphicx}
\usepackage[all,knot]{xy}

\pagestyle{fancy}
\lhead{Ben Carriel}
\chead{Math 4530 Final Exam}
\rhead{\today}

\parskip 7.2pt
\parindent 8pt

\DeclareMathOperator{\N}{\mathbb{N}}
\DeclareMathOperator{\Z}{\mathbb{Z}}
\DeclareMathOperator{\Q}{\mathbb{Q}}
\DeclareMathOperator{\R}{\mathbb{R}}
\DeclareMathOperator{\C}{\mathbb{C}}
\DeclareMathOperator{\capchi}{\raisebox{2pt}{$\mathlarger{\mathlarger{\chi}}$}}

\DeclareMathOperator{\divides}{\mathrel{|}}
\DeclareMathOperator{\suchthat}{\mathrel{|}}

\DeclareMathOperator{\lra}{\longrightarrow}
\DeclareMathOperator{\into}{\hookrightarrow}
\DeclareMathOperator{\onto}{\twoheadrightarrow}
\DeclareMathOperator{\bijection}{\leftrightarrow}

\newcommand{\problem}[1]{\noindent{\textbf{Problem #1}}\\}
\newcommand{\problempart}[1]{\noindent{\textbf{(#1)}}}
\newcommand{\exercise}[1]{\noindent{\textbf{Exercise #1:}}}

\newcommand{\der}[2]{\frac{\partial #1}{\partial #2}}
\newcommand{\norm}[1]{\|#1\|}
\newcommand{\diam}[1]{\text{diam}(#1)}
\newcommand{\seq}[2]{\{#1_{#2}\}_{#2 = 1}^\infty}

\DeclareMathOperator{\im}{\text{im}}

\newtheorem*{thm}{\\ Theorem}
\newtheorem*{lem}{\\ Lemma}
\newtheorem*{claim}{\\ Claim}
\newtheorem*{defn}{\\ Definition}
\newtheorem*{prop}{\\ Proposition}

\xyoption{arc}

\begin{document}

\message{ !name(4530final.tex) !offset(520) }
$\{f_n(x_k) \suchthat n \in C_{k-1}\}$. As before, this is
    a bounded sequence in $\R^k$ and has a convergent subsequence and
    we set $C_k$ to be the set of indicees of that convergent
    subsequence. So we have that we have that $f_n(x_k)$ exists as
    $n\to\infty$ within $C_k$. This completes the construction.
  \end{proof}
  Let $i_k$ be the $k^{\text{th}}$ term of $C_k$ in the preceeding
  lemma and set
  \[
  C = \{i_1, i_2, \ldots\}
  \]
  By construction, there are at most $k-1$ terms of $C$ that are not
  in $C_k$ and so $\lim_{n\to\infty} f_n(x)$ exists for every $x\in E$
  as $n\to\infty$ within $C$.

  Now choose an $\epsilon > 0$ and apply the equicontinuity of the
  $f_n$ to find a $\delta > 0$ such that $d(p,q) < \delta$ implies
  that $|f_n(p) - f_n(q)| < \epsilon$ for all $n$. Cover $X$ with a
  set of balls of radius $\delta/2$ centered at each point. Because
  $X$ is compact we can find a finite subcover $B_1, B_2,\ldots, B_M$
  of open balls. Because $E$ is dense in $X$ there are points $p_i \in
  B_i \cap E$ for $i \leq M$. Moreover, we see that $p_i \in E$ imples
  that $\lim_{n\to\infty} f_n(p_i)$ exists within $C$. Hence, we can
  find an integer $N$ large enough that $|f_m(p_i) - f_n(p_i)| <
  \epsilon$ for $i \leq M$ and $n,m > N$ and $n,m \in C$.

  Now we just use the triangle equality to get the uniform
  convergence. Pick an $x\in X$. then $x \in B_i$ for some $i$ and
  $d(x,p_i) < \delta$. If we choose $\delta$ and $N$ as above we have that
  
\message{ !name(4530final.tex) !offset(564) }

\end{document}